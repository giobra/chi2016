\documentclass{sigchi}

% Use this command to override the default ACM copyright statement
% (e.g. for preprints).  Consult the conference website for the
% camera-ready copyright statement.


%% EXAMPLE BEGIN -- HOW TO OVERRIDE THE DEFAULT COPYRIGHT STRIP -- (July 22, 2013 - Paul Baumann)
% \toappear{Permission to make digital or hard copies of all or part of this work for personal or classroom use is      granted without fee provided that copies are not made or distributed for profit or commercial advantage and that copies bear this notice and the full citation on the first page. Copyrights for components of this work owned by others than ACM must be honored. Abstracting with credit is permitted. To copy otherwise, or republish, to post on servers or to redistribute to lists, requires prior specific permission and/or a fee. Request permissions from permissions@acm.org. \\
% {\emph{CHI'14}}, April 26--May 1, 2014, Toronto, Canada. \\
% Copyright \copyright~2014 ACM ISBN/14/04...\$15.00. \\
% DOI string from ACM form confirmation}
%% EXAMPLE END -- HOW TO OVERRIDE THE DEFAULT COPYRIGHT STRIP -- (July 22, 2013 - Paul Baumann)


% Arabic page numbers for submission.  Remove this line to eliminate
% page numbers for the camera ready copy 

%\pagenumbering{arabic}

% Load basic packages
\usepackage{balance}  % to better equalize the last page
\usepackage{graphics} % for EPS, load graphicx instead 
%\usepackage[T1]{fontenc}
\usepackage{txfonts}
\usepackage{times}    % comment if you want LaTeX's default font
\usepackage[pdftex]{hyperref}
% \usepackage{url}      % llt: nicely formatted URLs
\usepackage{color}
\usepackage{textcomp}
\usepackage{booktabs}
\usepackage{ccicons}
\usepackage{todonotes}
\usepackage{soul}
\usepackage{paralist}

\usepackage{caption}
\usepackage{subcaption}
\usepackage[super]{nth}
\usepackage{amsmath}

% llt: Define a global style for URLs, rather that the default one
\makeatletter
\def\url@leostyle{%
  \@ifundefined{selectfont}{\def\UrlFont{\sf}}{\def\UrlFont{\small\bf\ttfamily}}}
\makeatother
\urlstyle{leo}

% To make various LaTeX processors do the right thing with page size.
\def\pprw{8.5in}
\def\pprh{11in}
\special{papersize=\pprw,\pprh}
\setlength{\paperwidth}{\pprw}
\setlength{\paperheight}{\pprh}
\setlength{\pdfpagewidth}{\pprw}
\setlength{\pdfpageheight}{\pprh}

% Make sure hyperref comes last of your loaded packages, to give it a
% fighting chance of not being over-written, since its job is to
% redefine many LaTeX commands.
\definecolor{linkColor}{RGB}{6,125,233}
\hypersetup{%
  pdftitle={SIGCHI Conference Proceedings Format},
  pdfauthor={LaTeX},
  pdfkeywords={SIGCHI, proceedings, archival format},
  bookmarksnumbered,
  pdfstartview={FitH},
  colorlinks,
  citecolor=black,
  filecolor=black,
  linkcolor=black,
  urlcolor=linkColor,
  breaklinks=true,
}

% create a shortcut to typeset table headings
% \newcommand\tabhead[1]{\small\textbf{#1}}

% End of preamble. Here it comes the document.
\begin{document}

\title{To be decided\\
Measuring Interaction before developing prototypes\\
Measuring Interaction Graphs}

\numberofauthors{3}
\author{%
  \alignauthor{1st Author Name\\
    \affaddr{Affiliation}\\
    \affaddr{City, Country}\\
    \email{e-mail address}}\\
  \alignauthor{2nd Author Name\\
    \affaddr{Affiliation}\\
    \affaddr{City, Country}\\
    \email{e-mail address}}\\
  \alignauthor{3rd Author Name\\
    \affaddr{Affiliation}\\
    \affaddr{City, Country}\\
    \email{e-mail address}}\\
}

\maketitle

\begin{abstract}
  Early development and testing of prototypes is good
  practice for user interface development. However, prototypes have
  to cover specific usage scenarios, and because they are limited
  in focus, the whole picture of the user interface is easily lost. Even
  simple questions dealing with numerosity and length of
  optimal execution paths or impact of possible user errors can be
  answered only for the specific scenarios being analysed.

  We discuss a tool that transforms models of the behaviour of a user
  interface into a graph. This is then used to specify usage
  scenarios, and to generate possible execution traces. Metrics based
  on number and length of different possible execution paths, with or
  without possible mistakes, can be easily computed. We apply the tool
  and metrics to well known examples of web mail apps, \hl{and show that
  important conclusions can be drawn even before the first prototype
  is built. --- \textbf{replace with a real example statement}}.
  
\end{abstract}

\keywords{Experimental; Evaluation; Statecharts: UML; UML-IDEA; Testing.}

\category{H.5.1}{Information interfaces and presentation (e.g., HCI)}{Multimedia Information Systems}. 
\category{H.5.2}{Information interfaces and presentation (e.g.,
  HCI)}{User Interfaces}. 

\section{Introduction}
We present an approach that allows a designer to quickly 
\begin{inparaenum}[(a)]
\item understand how prone an application is to user navigation errors;  
\item understand how recoverable the application is from those errors;
\item support task analysis and task / scenario design;
\item support interaction design (IxD) in terms of consistency, error-proneness, and recovery premature commitment;
\end{inparaenum}
and further, allows different designs to be objectively compared to support designers in evaluating interaction sequences. To explain our work we present a case study
comparing four different web mail front ends. We have chosen this
domain because it is very well understood by readers and yet those
trivial questions lead to non trivial results.

The approach is based on UML state-chart models of the user interfaces
which are automatically processed to produce \emph{interaction graphs}. These are then used to unfold \emph{execution traces} that
are dependent on the specific usage scenarios being considered in the
analysis. On traces several graph-theoretic computations can be
performed to produce a dashboard of different results that provide the
answers. Except for development of models and specification of the
desired scenarios, the other steps are totally automatic; models of a
design could be developed in a matter of a couple of hours.

Developing good user interfaces for web or mobile applications is a
complex and expensive endeavour. One reason is the combination of
devices, interaction modalities and workflows that need to be
supported.

Adopting Usage Centred Development practices is effective, as is
following established design principles~\cite{constantine99}. Early
prototyping~\cite{buxton07} to explore part of the five-dimensional
prototyping space~\cite{mccurdy06} is one of the most effective
techniques, especially when paired with usability investigations based
on user testing or heuristic evaluations.

However, it still requires development of prototypes that are usually
developed with certain tasks in mind, and therefore are quite
restricted in terms of depth and breadth of supported use
cases. Furthermore, usability results are always surrounded by a cloud
of uncertainty, due to subjectivity introduced by participants and
facilitators or by other contingency factors involved in the
analysis. Thus, although an effort needs to be expended to develop and
use prototypes, less than optimal results are obtained.

A designer, while conceiving and developing a solution, might benefit
from answers to seemingly simple questions that should not require a
significant investment of time and effort. Given one or more potential
solutions and some usage scenarios, interesting questions could
include: \hl{``What are the optimal execution paths?'' or ``How many such
paths exist?'', or ``Are usage mistakes recoverable?'' -- \textbf{these seem to be software `executionist' and not so much user/designer is -- maybe we can say something more designer and save these to the technical bit of the paper?}}. At the moment,
even these straightforward questions are quite complex to answer. In
fact, they require inspection of prototypes, manual tracking of which
screens and widgets are used at which stage, and exhaustive searches.

This should not be the case, however. These answers provide important
insights to a designer, and support decisions related to benchmarking
different solutions, to identification of optimal or mistaken paths,
to assessment of suitability of a design with respect to scenarios.

The contribution consists of the development of a tool that transforms
models and scenario specifications into execution traces, and the
definition of metrics that provide concise, precise and objective
measures of a design.

% REFERENCES FORMAT
% References must be the same font size as other body text.
\bibliographystyle{SIGCHI-Reference-Format}
\bibliography{gb,database}

\end{document}

%%% Local Variables:
%%% mode: latex
%%% TeX-master: t
%%% End:
