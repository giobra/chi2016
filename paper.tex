\documentclass{sigchi}

% Use this command to override the default ACM copyright statement
% (e.g. for preprints).  Consult the conference website for the
% camera-ready copyright statement.


%% EXAMPLE BEGIN -- HOW TO OVERRIDE THE DEFAULT COPYRIGHT STRIP -- (July 22, 2013 - Paul Baumann)
% \toappear{Permission to make digital or hard copies of all or part of this work for personal or classroom use is      granted without fee provided that copies are not made or distributed for profit or commercial advantage and that copies bear this notice and the full citation on the first page. Copyrights for components of this work owned by others than ACM must be honored. Abstracting with credit is permitted. To copy otherwise, or republish, to post on servers or to redistribute to lists, requires prior specific permission and/or a fee. Request permissions from permissions@acm.org. \\
% {\emph{CHI'14}}, April 26--May 1, 2014, Toronto, Canada. \\
% Copyright \copyright~2014 ACM ISBN/14/04...\$15.00. \\
% DOI string from ACM form confirmation}
%% EXAMPLE END -- HOW TO OVERRIDE THE DEFAULT COPYRIGHT STRIP -- (July 22, 2013 - Paul Baumann)


% Arabic page numbers for submission.  Remove this line to eliminate
% page numbers for the camera ready copy 

\pagenumbering{arabic}

% Load basic packages
\usepackage{balance}  % to better equalize the last page
\usepackage{graphics} % for EPS, load graphicx instead 
%\usepackage[T1]{fontenc}
\usepackage{txfonts}
\usepackage{times}    % comment if you want LaTeX's default font
\usepackage[pdftex]{hyperref}
% \usepackage{url}      % llt: nicely formatted URLs
\usepackage{color}
\usepackage{textcomp}
\usepackage{booktabs}
\usepackage{ccicons}
\usepackage{todonotes}
\newcommand{\mytodo}[1]{\todo[inline]{#1}}
\newcommand{\ie}{\emph{i.e.}}
\newcommand{\eg}{\emph{e.g.}}
\newcommand{\migtool}{MIGtool\ }
\usepackage{soul}
\usepackage{paralist}
\usepackage{listings}

\usepackage{caption}
\usepackage{subcaption}
\usepackage[super]{nth}
\usepackage{amsmath}

% llt: Define a global style for URLs, rather that the default one
\makeatletter
\def\url@leostyle{%
  \@ifundefined{selectfont}{\def\UrlFont{\sf}}{\def\UrlFont{\small\bf\ttfamily}}}
\makeatother
\urlstyle{leo}

% To make various LaTeX processors do the right thing with page size.
\def\pprw{8.5in}
\def\pprh{11in}
\special{papersize=\pprw,\pprh}
\setlength{\paperwidth}{\pprw}
\setlength{\paperheight}{\pprh}
\setlength{\pdfpagewidth}{\pprw}
\setlength{\pdfpageheight}{\pprh}

% Make sure hyperref comes last of your loaded packages, to give it a
% fighting chance of not being over-written, since its job is to
% redefine many LaTeX commands.
\definecolor{linkColor}{RGB}{6,125,233}
\hypersetup{%
  pdftitle={SIGCHI Conference Proceedings Format},
  pdfauthor={LaTeX},
  pdfkeywords={SIGCHI, proceedings, archival format},
  bookmarksnumbered,
  pdfstartview={FitH},
  colorlinks,
  citecolor=black,
  filecolor=black,
  linkcolor=black,
  urlcolor=linkColor,
  breaklinks=true,
}

% create a shortcut to typeset table headings
% \newcommand\tabhead[1]{\small\textbf{#1}}

% End of preamble. Here it comes the document.
\begin{document}

\title{Measuring Interaction Design}

\numberofauthors{2}
\author{%
  \alignauthor{1st Author Name\\
    \affaddr{Affiliation}\\
    \affaddr{City, Country}\\
    \email{e-mail address}}\\
  \alignauthor{2nd Author Name\\
    \affaddr{Affiliation}\\
    \affaddr{City, Country}\\
    \email{e-mail address}}\\
}

\maketitle

\begin{abstract}
  Early prototyping of user interfaces is an established good
  practice. However, because prototypes cover only some usage
  scenarios, the whole picture of the user interface is easily
  lost.  Questions dealing with number of required steps,
  possible interaction paths or impact of possible user errors can be
  answered only for the specific scenarios being analysed and after
  tedious manual inspection.

  We present a tool that transforms models of a user interface into a
  graph, upon which usage scenarios can be easily specified, and used
  to compute possible interaction paths. Metrics based on possible
  paths, with or without possible user errors, can be easily computed.
  For example, when applying them to mail applications, we learn that
  Gmail has 3 times more shortest routes, has 2x more routes that
  include one user error, has routes with 13\% fewer steps, but has
  the smallest probability to hit an optimal route.

  
\end{abstract}

\keywords{Experimental; Evaluation; Statecharts: UML; Metrics; Testing.}

\category{H.5.1}{Information interfaces and presentation (e.g., HCI)}{Multimedia Information Systems}. 
\category{H.5.2}{Information interfaces and presentation (e.g.,
  HCI)}{User Interfaces}. 

\section{Introduction}

\begin{quote}
\mytodo{ Subcommittee: Technology, Systems, and Engineering

People we should pay particular attention to:
Caroline Appert
Conversy, St�phane
Nebeling, Michael

Keyword we need to pay attention to:
This subcommittee will focus on technology, systems and engineering
contributions that enable, improve, or advance interaction. This will
include software and hardware technologies and systems that enable and
demonstrate novel interactive capabilities, as well as languages,
methods and tools for \hl{CONSTRUCTION AND ENGINEERING} of interactive
systems. Engineering contributions should clearly demonstrate how they
address interactive systems concerns such as, for example,
scalability, reliability, interoperability, testing, and
performance. Systems and technology contributions will be judged by
their technical innovation and/or ability to connect, SIMPLIFY or
enrich interactions, for example in intelligent interfaces and
mobile/ubiquitous computing.
}

\end{quote}


We present an approach that allows a designer to assess interaction
design (IxD) qualities such as efficiency,
error-proneness and recovery from premature committment. Key
importance is given to the ability to quickly
\begin{inparaenum}[(a)]
\item understand how supportive a user interface (UI) is with respect
  to user efficiency;
\item understand how prone the UI is to user navigation
  errors;
\item understand how recoverable the UI is from those errors; and,
\item perform objective quantitative comparison of different designs
  to support construction and engineering of interactive systems.
\end{inparaenum}

To explain our work we present a comparison of four web mail front
ends. We have chosen this domain because it is easy to understand and
yet several trivial questions are not easily answered. We applied
the same techniques also in other domains, such as HVAC and other
embedded UIs.


Developing good UIs for web or mobile applications is a
complex and expensive endeavour. One reason is the combination of
devices, interaction modalities and workflows that need to be
supported.
%
Adopting Usage Centered Development (UCD) practices is effective, as is
following established design principles~\cite{constantine99}. In
particular, one of the most effective technique is early
prototyping to explore part of the five-dimensional fidelity
prototyping space~\cite{mccurdy06,buxton07}. It is particularly effective when
paired with usability investigations based on user testing or
heuristic evaluations~\cite{preece02,rubin08}.

However, UCD requires prototypes which are usually developed with
certain tasks in mind, and therefore are quite restricted in terms of
depth, breadth, dynamics and data. Furthermore, in addition to the
possible bias introduced by prototypes, usability results are always
surrounded by a cloud of uncertainty, due to subjectivity introduced
by participants and facilitators or by other contingency factors
involved in the analysis. Thus, although a  significant effort
needs to be directed to develop and use prototypes, less than optimal
results are obtained. 

Even worse, there might be questions that cannot be easily
answered. Given one or more potential designs and some usage
scenarios, interesting questions could include: ``How many different
routes can be followed by the user to carry out the scenario?'',
``Which are the shortest ones?'', ``If a user makes a mistake, would
he or she be able to recover?'', ``How many steps would the recovery
require?''. In a different setting, when designing and evaluating
embedded UIs (such as when dealing with plane's
cockpits~\cite{conversy07}), other relevant questions might include
``How would the above properties change if we add a certain a
widget?'', or ``... if we replace a widget with another?''.  At the
moment, these straightforward questions are quite complicated to
answer. In fact, they require development of prototypes, inspecting
them, manual tracking of which screens and widgets are used at which
stage, and a systematic manual analysis.

This should not be the case, however. Answers to these questions could
be automatically computed. In this way a system could provide
important insights to a designer, and  support assements of
potential user flexibility, user efficiency, error proneness, ability
to recover, compactness and consistency of a design.

Our approach is based on UML statechart models of UIs which are
automatically processed to produce \emph{interaction graphs}. These
graphs are then used to specify interaction scenarios and to unfold
the possible interaction paths (called \emph{execution traces})
determined by the specific scenarios being considered. Traces are fed
to several graph-theoretic computations which produce a dashboard with
different results. Except for development of models and specification
of the desired scenarios, which have to be done manually, the other
steps are totally automatic; models of a design (such as the ones
shown below) can be developed in a matter of a couple of hours.

Our contribution consists of
\begin{inparaenum}[(a)]
\item the idea of using a model to specify interaction scenarios,
\item the development of a tool (MIG-TOOL, Measuring Interaction
  Graphs Tool) that transforms models and scenario specifications into
  interaction paths, and
\item the definition of metrics that provide concise, precise and
  objective measures of a design.
\end{inparaenum}
The examples included in the paper show that among four web mail
applications, and with respect to a typical usage scenario, Gmail is
the most efficient and flexible UI, with the best ability to recover
from user errors, but only for users that possess a certain level of
proficiency. In fact, Gmail has the largest number of shortest paths
(when users are supposed to make 0 or 1 error); when users make 2 or
more errors the number of paths drops significantly, reducing the
error-proneness of the UI; on the best case, Gmail features also the
shortest paths, requiring 13\% fewer steps; however, the probability
that a user hits an optimal path with Gmail is 10 times smaller than
the best of the other applications, and the probability that a random
walker hits a state that is not due to an error is 10\% smaller than
the best of the other applications.
%
These values suggest that Gmail offers more ways to accomplish tasks
included in the scenario, that comparably more of these ways do not
involve extra steps, that they require fewer steps, and that it might
be more difficult for novice users to exploit the most efficient
ways. Further inspections show that some differences are due to the
slightly different interaction structures adopted for uploading
messages. If Gmail didn't exist yet, by using \migtool its designers
could obtain these answers well before developing prototypes and
performing usability studies.

\section{Background}
\label{sec:background}

The literature on using state-transition network representations for
specifying or analysing the behavior of UIs is vast. We conducted a
systematic-style literature review, using Google Scholar and queries
with combinations of these phrases: ``user interface'', ``path
analysis'', ``user trace'', ``interaction trace'', ``navigation'',
``markov chain'', ``state transition'', ``statechart'', ``metric'',
``measure''. For each query we analyzed title and abstract of the
first 50 hits and appraised their relevance based on whether the paper
discusses approaches for measuring user interfaces in the context of
usability and on the basis of a model based on state transitions. This
resulted in 78 full-text papers that were later on re-analyzed against
the same criterion, leading to several of papers mentioned below.
%
Several of the examined papers deal with testing user interfaces, and
problems related to generating execution traces as a means to assess
coverage of a test
suite~\cite{ammann2008introduction,briand2005:test:uml}. They were
excluded from the review.

Usage of state-transition networks to model UI behavior in order to
draw usability conclusions dates back at least to~\cite{parnas69}. In
it, Parnas claimed that several kinds of usability problems would not
occur if the designer adopted a design framework where states and
their transitions are made explicit.

In many cases \emph{statecharts} are used, a generalization of finite
state automata, which is a more expressive language.
%
Horrocks showed how statecharts can be  used to model and
specify the dynamics of typical desktop UIs~\cite{horrocks99}. While
providing many interesting insights on how and why one should use
statecharts to do so, this nice work does not address how such a
specification could be \emph{automatically} processed.
%
This idea was later on expanded by
Thimbleby~\cite{thimbleby07} , which discusses the use of statecharts
as meta-models for user interfaces; they are seen as crucial
representations that allow a designer to fully appreciate how devices
behave. The overall claim is that ``If you don't understand the logic
conveyed by a statechart model of a user interface, then you don't
understand the behavior of the user interface''.

WebML~\cite{ceri:00} is one of the most successful model-driven
approaches to web development (UIs and backend systems), with
industrial traction and a large number of publications. The language
is based on state transitions and is targeted to automatic generation
of data-intensive web applications.  %WebML encompasses different
%meta-models, independent from the technologies used to deliver the
%final application. % The WebML
% meta-model provides relatively few primitive units that can be
% combined to generate quite complex interactions; the modeling language
% is thus powerful and concise.  
Many other similar approaches involve or are based on task or activity
models~\cite{mori02,silva03:umli,koch02:uwe,gomez01,melia08}.
%
A recent OMG standard, called Interaction Flow Modeling Language
(IFML)~\cite{ifml13}, derives from WebML and focuses specifically on
user interaction. IFML is a language for specifying the structure of a
user interface and its behavior. It offers most of the abstractions
that are available in statecharts, mixed with the ability to specify
so-called ``components'' that are used to display and manipulate data
(to display details of a item, to display or select lists of items, to
input an item). 

A rather different route for the problem of generating UIs is followed
in~\cite{heymann2007formal}. They assume that the UI to be generated
is used to supervise and to monitor an underlying machine (\eg,
autopilot of a plane) which is modeled as a statechart. After assuming
that the behavior of the UI can be modeled as a statechart, they
devised an algorithm that checks whether the two models are
compatible, and that refines the UI model so that its states and
transitions are minimized while still allowing a correct manipulation
of the underlying machine. Application of such a technique leads to
UIs that are correct by-construction.

In~\cite{conversy07} the Interactive Cooperative Objects formalism is
discussed, which is based on Petri nets, yet another state transition
formalism.  It is used to enrich the ARINC 661 specification of
Cockpit Display System used in interactive cockpits of airplanes, so
that the semantics of widgets can be expressed and the behavior of a
UI be formally analyzed.

%  that are now under provides precise information
% for communication protocol between application (called User
% Applications) and user interface components (called widgets) as well
% as precise information about the widgets themselves. However, in ARINC
% 661, no information is given about the behaviour of these widgets and
% about the behaviour of an application made up of a set of such
% widgets.  The Interactive Cooperative Objects (ICOs) formalism is a
% formal description technique dedicated to the specification of
% interactive systems [4, 11]. It uses concepts borrowed from the
% object-oriented approach to describe the structural or static aspects
% of systems, and uses high-level Petri nets [8] to describe their
% dynamic or behavioural aspects.

Finite state representations have been used also as a conceptual
framework to write the code of widgets so that events and event
handlers in the UI can more easily be conceived, developed and
verified (\eg, \cite{appert2008swingstates}). 


% in java, FSA used as a conceptual framework to write the code of
% widgets so that events and event handlers in the ui can more easily be
% conceived, developed and verified.

% They say: StateCharts were used in the StateMaster User Interface
% Management System [40], and a variant of them specifically tailored to
% designing user interfaces was used in the more recent HsmTk toolkit
% [18]. StateCharts however are significantly more complicated and hard
% to learn than plain state machines, and our experience is that user
% interface designers and developers have difficulties exploiting their
% power. Other approaches include Petri Nets [41], which have also been
% used to specify user interfaces, for example in the PetShop system
% [42]. Here too, the learning curve is steep, making the adoption of
% such a model by developers difficult.

% With SwingStates, any number of state machines can run simulateneously. A state machine
% can be active, i.e., handling the events it receives, or inactive,
% i.e., ignoring events

% We adopt a similar apporach, but using UML state machine (which are
% more powerful than FSA) for conceiving, guiding development, analysis
% and verification of the behavior of the entire app.

% Nebeling says:\cite{nebeling2011metrics}

% Despite the fact that screen sizes and average screen resolutions have
% dramatically increased over the past few years, little attention has
% been paid to the design of web sites for large, high-resolution
% displays that are now becoming increasingly used both in enterprise
% and consumer spaces. We present a study of how the visual area of the
% browser window is currently utilised by news web sites at different
% widescreen resolutions. The analysis includes measurements of space
% taken up by the article content, embedded ads and the remaining
% components as they appear in the viewport of the web browser. The
% results show that the spatial distribution of page elements does not
% scale well with larger viewing sizes, which leads to an increasing
% amount of unused screen real estate and unnecessary scrolling. We
% derive a number of device-sensitive metrics to measure the quality of
% web page layout in different viewing contexts, which can guide the
% design of flexible layout templates that scale effectively on large
% screens.

% The purpose of ARINC 661 specification (ARINC 661, 2002) is to define
% interfaces to a Cockpit Display System (CDS) used in interactive
% cockpits that are now under

% application of a formal description technique to the various elements
% of ARINC 661 specification within an industrial project. This formal
% description technique called Interactive Cooperative Objects defines
% in a precise and non-ambiguous way all the elements of ARINC 661
% specification. The application of the formal description techniques is
% shown on an interactive application called MPIA (Multi Purpose
% Interactive Application). Within this application, we present how ICO
% are used for describing interactive widgets, User Applications and
% User Interface servers (in charge of interaction techniques). The
% emphasis is put on the model-based management of the feel of the
% applications allowing rapid prototyping of the external presentation
% and the interaction techniques.

% The Interactive Cooperative Objects (ICOs) formalism is a formal
% description technique dedicated to the specification of interactive
% systems [4, 11]. It uses concepts borrowed from the object-oriented
% approach to describe the structural or static aspects of systems, and
% uses high-level Petri nets [8] to describe their dynamic or
% behavioural aspects.

% The Interactive Cooperative Objects (ICOs) formalism is a formal
% description technique dedicated to the specification of interactive
% systems [4, 11]. It uses concepts borrowed from the object-oriented
% approach to describe the structural or static aspects of systems, and
% uses high-level Petri nets [8] to describe their dynamic or
% behavioural aspects.

The issue of measuring UI/interaction properties based on state
transition representations is not so well covered in the literature. A
review of usability measuring practices~\cite{hornbaek2006current},
under the headings \emph{measures of efficiency/usage patterns}, lists
only number of keystrokes, mouse clicks, and visited objects as
possible metrics. The notion of \emph{deviation from optimal solution}
is discussed only in the context of tools for route planning in 3D
navigation. In this paper we provide our own definition of deviation,
which applies to interaction with the UI; we provide also our notion
of execution traces across states of the UI, and the length of these
traces can be used as a measure of efficiency.

Lostness~\cite{smith1996lostness,otter2000lostness} is one of the few
metrics that were defined to measure the degree to which users become
lost in the information space, and therefore considers the notion of
\emph{deviation}. Defined specifically for hypertext systems, lostness
is a user performance measure that considers among others the number
of visited nodes, the number of different nodes that were visited
nodes, and the number of required nodes. This measure of efficiency is
usually applied to traces of actual users, and is argued to be
suitable for hypertext systems because the predominant task is
browsing information, rather than trying to achieve specific goals. It
therefore relaxes three assumptions: that there is a task to complete,
that there is a correct way to carry it out, that the purpose of the
system is to support users to carry out their tasks. The lostness
metric can be easily computed on top of the data that we use with our
approach; it would be yet another metric that can be provided to
assess quality of a UI. Notice that in our case we do not need to user
data to compute it.

A usability analysis method capable to analytically predict task
completion times from a storyboard of the UI is
CogTool~\cite{bellamy2011cogtool}. CogTool relies on the ACT-R
cognitive modeling engine, and allows a designer to setup a
low-fidelity prototype of the UI. After deciding which interaction
modality and which widgets are used to implement actions, the designer
gets a validated estimation of how long an experienced user would
spend on each step. CogTool takes care of adding extra ``mental''
steps, such as ``think'' steps, before certain patterns of provided
steps, according to the cognitive theory underlying ACT-R. As a
result, users of CogTool obtain the breakdown of the times required by
each of the steps. Our method is weaker in terms of cognitive validity
and in terms of precision of the output: it does not provide expected
completion times. However, with our method one can analyze a large
part of the UI, get information about possible problems in some areas,
and only then devolve more resources in building storyboards and
in making assumptions regarding widgets so that specific execution
paths previously identified can be analyzed with CogTool. In a sense,
the output of our method could be used to make informed decisions as
to what to analyze with CogTool.

Markov models, \ie\ directed graphs where edges leaving a vertex are
associated to a probability distribution, were used
in~\cite{thimbleby2001markov}, as a means to perform usability
analysis as early as possible, even before building prototypes of the
UI. Vertices represent states of the UI and edges correspond to user
actions (such as pressing a push-down button).  Probabilities can be
used as a model of the skills or knowledge of the user: equiprobable
actions correspond to a knowledge-free user, whereas when some actions
have a very low probability it means that for that user the action is
unlikely to be executed (perhaps because little known). Simple
mathematical operations on the transition matrix of the model give the
probability that after $n$ steps from a given initial state, the UI is
in a given state. With Markov models, by manipulating probabilities,
the designer can plot the number of required steps as a function of
how close the probabilities are to the designer's ``perfect''
knowledge.  Examples discussed in the paper cover several devices,
ranging from a simple torch (with 4 states), a microwave cooker (6
states), a mobile phone (152 states), but all being push-down devices
with a fixed set of buttons. This is obviously not the case for UIs of
information systems, where some buttons may or may not be ``painted''
on the screen. This makes it more difficult to specify the transition
matrix. Another limitation of this approach is that probabilities are
assumed to be constant, corresponding to users that are not learning
while interacting with the system.
%
Our approach is based on statecharts, a language that in practice is
more expressive than Markov models, making it easier to specify the
UI behavior, especially in cases where the set of buttons change over
time. While our examples do not make use of probabilities, this is
very easy to cope with (see the Discussion at the end of the paper for
some of the benefits that doing this could bring). Similarly
to~\cite{thimbleby2001markov}, our approach could be used when
conservative results that do not rely on psychological assumptions are
sought. 


A discussion of social network analysis metrics applied to interaction
design is provided in~\cite{thimbleby09:sna:eics09}. Once more, a UI
is modeled in terms of directed graphs (vertices are states and edges
are actions), and various centrality metrics are used to draw
conclusions that bear upon usability. For example, centrality measures
(such as the Sabidussi,  Jordan, or betweenness) can be used to
identify states that are good places to start from to get to other
states. Other metrics, such as edge betweenness, can be used to
identify actions that are important because most of the shortest
paths go through these actions. The paper presents compelling examples
of using this technique to identify shortcomings in the design of
infusion pumps. 
%
In our work we automatically generate graphs from statechart models,
and tried to use these metrics. We were not able to draw sensible
conclusions from the values we obtained (for example from the models
presented below). One possible explanation may rest on the different
types of models: in our case they reflect the variety and flexibility
with which ``buttons'' can be used in modern web applications. For
infusion pumps the UI is more constrained in how a task can be carried
on. 

vedere \cite{hilbert2000}
\mytodo{However, because user interface
  events are typically voluminos and rich in detail, automated support
  is generally required to extract information at a level of
  abstraction that is useful to investigators interested in analyzing
  application usage or evaluating usability. This survey examines
  computer-aided techniques used by HCI practitioners and researchers
  to extract usability-related information from user interface
  events. A framework is presented to help HCI practitioners and
  researchers categorize and compare the approaches that have been, or
  might fruitfully be, applied to this problem.

  They - mention levels of abstractions in user events; what we deal
  with here are ``abstract interaction level'' events - they mention
  the problem of detecting sequences from event streams, which in a
  sense is the converse of what we do - they discuss partial matches
  between event streams and ``target'' sequences; we do the converse:
  generation of event streams with detours from target sequences.  }

% Voida\cite{voida} says:
% We describe an alternative model for organizing the
% desktop interface-activity-based computing-and
% identify a series of high-level system requirements for
% interfaces that use activity as their primary organizing
% principle.

% We could mention this work when we introduce scenarios, and say that
% people inventing these new metaphors could take advantage of our
% metrics if they model 2+ apps. 

% Eg. they say: Requirement 2. Activity-based systems should provide
% lightweight mechanisms to create, change, and alter
% activities, since heavyweight interaction techniques are
% likely to deter adoption and use.

% eg Requirement 7. Because information sharing is a
% �common case� in knowledge work, lightweight sharing
% capabilities should be integrated directly as a first-class
% interaction technique. 

\section{Generation of interaction traces}
\label{sec:traces}

Interaction traces are paths (\ie, sequences of connected states
of the model) that users can follow to perform a given scenario.  The
generation process includes the following steps:
\begin{inparaenum}[(1)]
\item automatic flattening of the statechart model;
\item manual definition of the interaction scenario;
\item automatic generation of execution traces; and
\item manual interactive analysis of results.
\end{inparaenum}

\subsection{Processing models }

\migtool\ takes as input UML statecharts, which are a
generalization of finite state automata (FSA), with an expressive language
that includes hierarchic levels of abstraction, concurrent regions,
states and pseudostates, an extended state notion based on an
arbitrary computational model.  Harel, the inventor of statecharts,
gives an interesting retrospective view of how they were invented,
back in early eighties, and why they were appealing also to non
experts~\cite{harel09}.  For the sake of brevity, we refrain from
describing them here in detail; the interested reader is referred to
the UML standard~\cite{uml25} or some of the textbooks that deal with
them, like~\cite{samek09}.

Using statecharts, behavior of UIs can be represented by associating
states to screens and particular configuration of widgets, and
transitions to actions performed by users or by the system itself. 

Because statechart models take advantage of abstraction features and a
rich set of connecting pseudostates, they are not suitable to be
directly processed to compute metrics. For this reason, \migtool first
\emph{flattens} the model. Flattening is a process often used when
statecharts have to be automatically
processed~\cite{briand2005:test:uml}, and it means to produce a FSA
that is behaviorally equivalent to the original statechart, where
hierarchy between states and concurrent regions are removed. In most
cases this leads to an exponential number of states and transitions,
but because the process is completely automatic and in our case there
is no need to manually inspect the resulting FSA, this aspect is in
many cases not relevant.  Flattening statecharts is an often used
process when they have to be automatically
processed~\cite{briand2005:test:uml}.  \migtool\ produces an XML
representation (graphML) of the resulting FSA, the \emph{interaction
  graph}. It is a directed multigraph, potentially with cycles and
loops, with edges that are labeled with the name of the corresponding
action\footnote{A directed multigraph is a graph such that there is at
  least a pair of vertices that are connected by 2 or more edges that
  have the same direction. Cycles are paths that include 2 or more
  occurrences of the same edge. Loops are edges that start and end on
  the same vertex.}. In the following we will be referring to the
interaction graph, and will be using as synonymous the terms ``state''
and ``vertex'', and ``action'' and ``edge''.

\subsection{Defining usage scenarios}

Because in all but the most trivial interaction graphs there are
cycles, the set of possible interaction traces is usually
infinite. For this reason, scenarios need to be defined as constraints
on the possible execution traces that can be generated.  Users of
\migtool\ define interaction scenarios by specifying key steps (called
\emph{bridge sets}) that users are expected to go through; we call
this process \emph{grounding usage scenarios on models}. Each bridge
set is specified by selecting a subset of the edges of the interaction
graph. Often a scenario might require also optional steps. In such a
case, the specification includes an initial state and a sequence of
bridge sets. For example, to specify a scenario for replying to an
email message, one could select all the edges associated to the action
``reply'' (bridge set 1), followed by edges labeled with ``typeBody''
(bridge set 2), followed by edges labeled with ``send'' (bridge set
3). A well formed bridge set would consist of 1 or more edges (a
bridge set with 0 edges would make the scenario unviable). In this way
scenarios with cycles can be formulated (\eg, reply to two
messages). Each pair of consecutive bridge sets makes a \emph{stage}
of the scenario.
%
To cope with multi edges, the interaction graph is \emph{simplified}:
all edges between  pairs of vertices are merged into a single one,
whose label includes the original ones.

\subsection{Searching traces}

Quite expressive languages can be conceived for grounding scenarios
(\eg, languages based on regular expressions). But such expressivity
bonus needs to be balanced with computational tractability: even
models with 50 states might lead to interaction graphs with several
hundred states and several thousand edges, leading to an enormous
number of possible paths to filter even for scenarios with just a few
stages. 

To cope with this we implemented a path searching algorithm that
processes each of the stages sequentially, starting from the initial
initial state. Given a bridge set $B_i$, the algorithm does a
breadth-first search of all the geodesic paths\footnote{``Geodesic''
  is synonymous with ``shortest''.}  that connect each of the ending
vertices of edges in the bridge set $B_i$ with some of the starting
vertices of edges in bridge set $B_{i+1}$. If some of the starting
vertices cannot be reached, then the bridge is dropped. \migtool\
creates a new graph from the geodesic paths found for each stage, and
then joins these graphs so that geodesic paths found for stage $i$ are
joined with those of ${i+1}$. These global paths, connecting the
initial state and the remaining bridges of the last bridge set of the
scenario are called \emph{execution traces}.

Notice that a scenario specifies only the desired occurrences of
actions, not all the necessary ones. For example, if the model
prescribes that in order to \texttt{view(message)} while reading
another one, one has to \texttt{goBack} to the inbox first, a scenario
specifying two consecutive \texttt{view(message)} would lead to
geodesic paths that include also the \texttt{goBack} action, even if
it is not specified in the scenario. It is the task of \migtool to
unfold cyles in the graph and search all the geodesic paths that
connect the desired user actions.

\subsection{Detour traces}
The algorithm described so far finds (some of) the global paths
connecting the initial state to one or more bridges for each of the
specified stages. The globally shortest path is identified, together
with other viable alternatives. These traces are called
\emph{detour order 0} traces (and states).

For each stage, up to a maximum detour order $H$, the search algorithm
creates traces with detour order $k+1$ by collecting states $D_k$
with order $0$ up to $k$, and by identifying the neighbours ($N_k$)
of $D_k$ (which are the states not included in $D_k$ that are
connected through an edge to some state in $D_k$). Edges connecting
$D_k$ with $N_k$ represent deviations that users might follow, and
geodesic paths between $N_k$ and $D_k$ constitute recovery paths that
users might follow to complete the scenario from states in
$N_k$. These deviations and recovery paths are joined and constitute
the traces of order $k+1$.  It could happen that
for some state in $N_k$ there is no path leading to any vertex in
$D_k$: in such a case the deviation leads to a dead end that prevents
the user to complete the scenario.

In general, \migtool is used to process a model against a scenario
and to generate execution traces of detour order = 0,
..., H. The time complexity of the search algorithm is
$\mathcal{O}(KM(E+V)))$, where there are $K$ bridge sets, the mean
size of them is $M$, the interaction graph has $V$ vertices and $E$
edges. Therefore it scales pretty well with the size of the
interaction graph and/or complexity of the scenarios. In practice for
interaction graphs consisting of about 10000 edges and a dozen of
bridge sets, on a low-cost PC it takes about 15 seconds to generate
traces of order 0 to 3.


\section{Comparing applications}
\label{sec:casestudy}

In this section we describe some examples based on well known web mail
clients, namely Gmail, Horde, SquirrelMail and Roundcube. We chose
these examples because they are very well known, and therefore are
easy to describe and understand. And yet, despite email being a very
well understood domain, the kind of questions that can be posed and
the answers that are found provide an insight on some of the usability
properties of these applications. 

\subsection{Models and scenarios}
\label{sec:models-scenarios}

Models of the four applications have been manually developed using UML
state machines. In order to support a fair comparison, all four models
cover the same set of use cases: listing the content of the inbox,
reading a message or conversation, replying to a message, composing
and sending a new message.

Figure~\ref{fig:model} shows part of the model of Gmail. At some
point, the user might be viewing all the conversations of the inbox
(state \texttt{viewingConversations}); available actions include
moving to the next or previuous block of conversations, refreshing the
list, or opening a specific conversation (transition
\texttt{open(conversation)}). This transition (which is assumed to
occur when the user clicks on one of the visible conversations) leads
to the state \texttt{viewingAConversation}, where the behavior of the
system is defined by a more detailed state machine. By default the
user is viewing a conversation, but by performing the \texttt{reply}
action the UI moves to a state called \texttt{replying}, where the
body of the message can be typed, the subject can be changed, or
another addressee can be added.

Notice that this behavior ``happens'' in one of the two concurrent
regions specified by this model. In parallel to this, the user can
either be reading messages (state \texttt{reading}) or may be
composing a new message (state \texttt{composing}). In the latter case
the user can independently add recipients, attachments, write the body
or subject of the message; and send or cancel it.

\begin{figure*}[tbh]
  \centering
  \includegraphics[width=\linewidth]{figures/gmail_2.png}
  \caption{Part of the Gmail model.}\label{fig:model}
\end{figure*}

The model that we show here is part of what we used in the examples
reported below, and that is only a part of what the real Gmail
application supports. The actual model that we used consists of 
24 states, 12 pseudo states, 12 regions, and 61 transitions. 

The flattening process, which takes a fraction of a second on a
low-cost PC, produces a directed multigraph with 47 vertices and 634
edges; when simplified by collapsing multiple edges between any pair of
vertices, the graph includes 312 edges. Each vertex represents one of
the possible combinations of simple states in any of the regions that
can be active at the same time.

Similar models and corresponding graphs were produced for the other
three applications. 



\subsection{Analysis of interaction designs}
\label{sec:analysis}

Inspection of the interaction graph is not particularly useful because
even for small graphs like the one  obtained from our Gmail model
no particular structure is evident. Figure~\ref{fig:gom} shows a plot
using a circular layout of the 47 states: because of the large
number of cycles that exist among states, edges form an intricated web
of possible action sequences. This in practice greatly reduces usefulness of
typical network analysis metrics, such as \emph{betweenness,
  eccentricity, page-rank, eigenvalue} centrality measures. 

\begin{figure}
  \centering
  \includegraphics[width=\linewidth]{figures/gom.png}
  \caption{Graph obtained from the Gmail model.}
  \label{fig:gom}
\end{figure}


To be able to obtain results that bear upon usability, we process
further the graph, first by specifying an interaction scenario using
vertices and edges of the graph, and secondly by computing possible
execution paths.

Specification of a scenario consists of formulating it using the
language provided by the graph, i.e. deciding which is the initial
state and which user actions need to be considered. For example, the
following fragment of R code specifies that the scenario should
include, in the given order, two occurrences of
\texttt{open(conversation)}, followed by \texttt{reply}, followed by
\texttt{typeBody}, etc.: \lstset{language=R}
\begin{lstlisting}
edgesWithLabel(gmail,"open(conversation)"),
edgesWithLabel(gmail,"open(conversation)"), 
edgesWithLabel(gmail,"reply"),
edgesWithLabel(gmail,"typeBody"),
edgesWithLabel(gmail,"send"),
edgesWithLabel(gmail,"compose"),
edgesWithLabel(gmail,"addRecipient"),
edgesWithLabel(gmail,"open(files)"),
edgesWithLabel(gmail,"writeBody"),
edgesWithLabel(gmail,"writeSubject"),
edgesWithLabel(gmail,"send")
\end{lstlisting}
In plain language such a scenario means opening one and then a second conversation,
replying to the last message of the second one by typing the body of
the response, sending it, and then composing a new message by adding a
recipient, an attachment, typing the body and subject, and finally
sending it.

The execution traces for such a Gmail scenario, with a detour limit of
3, consist of a graph with 474 vertices and 1753 edges. These traces
entail 12 possible geodesic paths with order 0, 82 with order 1, 30
with order 2 and 33 with order 3. Figure~\ref{fig:numpaths} shows the
number of paths obtained for the four applications, split by detour
order.


\begin{figure}[tbh]
  \centering
  \includegraphics[width=\linewidth]{figures/gpd.png}
  \caption{Number of geodesic paths split by detour order.}
  \label{fig:numpaths}
\end{figure}

% tot.hom
% ##     N min max        M       rel  N
% ## d0  4  15  18 15.75000 0.2539683  4
% ## d1 32  16  22 17.94531 1.7831955 36
% ## d2 18  18  24 19.25000 0.9350649 54
% ## d3 29  19  24 19.95690 1.4531317 83
%  tot.gom
% ##     N min max        M       rel   N
% ## d0 12  13  17 15.08333 0.7955801  12
% ## d1 82  14  24 19.11179 4.2905456  94
% ## d2 30  16  25 20.57778 1.4578834 124
% ## d3 33  17  26 21.51515 1.5338028 157
%  tot.sqm
% ##     N min max        M       rel   N
% ## d0  4  15  19 17.00000 0.2352941   4
% ## d1 44  16  24 19.77273 2.2252874  48
% ## d2 36  18  26 20.88889 1.7234043  84
% ## d3 56  19  25 21.64286 2.5874587 140
%  tot.rcm
% ##     N min max        M       rel  N
% ## d0  2  15  18 16.50000 0.1212121  2
% ## d1 27  16  24 19.07407 1.4155340 29
% ## d2 22  18  25 20.18182 1.0900901 51
% ## d3 31  19  24 21.04839 1.4727969 82

Gmail has the highest number of order 0 and 1 paths, and the highest
difference between the number of order 0 and 1, and between order 1
and 2 (at least 3 times as many order 0 paths than any of the other
applications, and at least twice as many order 1 paths as the any of
the other ones). 
 Gmail offers 3 times as many error-free alternative
paths to accomplish the scenario, which indicates that users might
more easily follow one of those paths. 
\mytodo{result}
However, Gmail offers also
almost twice as many order 1 paths, which means that users could be
more easily induced into an erroneous path than when using another
system.  
\mytodo{result}
Because the number of order 2 or 3 paths decreases, Gmail
reduces therefore the ``error proneness'' of this UI, for executions
that involve 2 or 3 errors. Notice that Roundcube has the smallest
number of order 0 paths (2 of them), which means that users are not
given much flexibility and freedom in carrying out correctly the
scenario. 
\mytodo{result}
For none of the systems a detour leads to dead-ends preventing the
user to complete the scenario.

Figure~\ref{fig:bestcase} shows the length of paths in the best case,
i.e., when users would always choose the shortest route. Gmail offers
the shortest paths across the four detour orders (for order 0 the
length is 13 steps, saving more than 13\% steps compared to other
applications; for order 3 the length is 17 steps; the other
applications are remarkably similar among them).  A plausible
interpretation is therefore that Gmail not only offers many more
error-free paths, but also gives the shortest ones. Users are given
more flexibility and more efficiency.  \mytodo{result} Because also
paths with order 1 or more are the shortest ones among the four
applications, Gmail also makes users more efficient in recovering from
errors.  \mytodo{result}

\begin{figure}[htb]
  \centering
  \includegraphics[width=\linewidth]{figures/min-paths.png}
  \caption{Minimum length of  geodesic paths split by detour order.}
  \label{fig:bestcase}
\end{figure}

%Figure~\ref{fig:averagecase} 


% \begin{figure}[htb]
%   \centering
%   \includegraphics[width=\linewidth]{figures/mean-paths.png}
%   \caption{Average length of  geodesic paths split by detour order.}
%   \label{fig:averagecase}
% \end{figure}

\begin{figure}
  \centering
  \includegraphics[width=\linewidth]{figures/prob-opt-path.png}
  \caption{Frequency of an optimal path.}
  \label{fig:freqopt}
\end{figure}

To combine these two results, we can easily compute the frequency of
paths with different length. Figure~\ref{fig:freqopt} shows, for each
application, the frequency of order 0 paths, the frequency of paths
with length less or equal to 15 (the minimum length across the four
applications), and the frequency of optimal paths (the shortest
ones). These values show that Gmail users have the lowest probability
to hit an order 0 path (because of the relative large number of order
1 paths made available by Gmail), have the highest probability to hit
a path with length 15 or less, have the lowest probability of hitting
the shortest paths (10 times smaller than the best of the other
applications).  Thus, the flexibility and efficiency that can be
exploited with Gmail are counterbalanced by the required knowledge and
capability of chosing an optimal path. In particular, Gmail offers
many detours of order 1 which increase flexibility for some users and
might decrease effectiveness for less knowledgeable ones.
\mytodo{result}

% >   d=0.050 # 5% chances that a user makes a generic mistake
% >   dprh=detour.page.rank(sc.hom,3,d);dprh
%          0          1          2          3 
% 0.73455373 0.11735984 0.05684419 0.09124223 
% >   dprg=detour.page.rank(sc.gom,3,d);dprg
%          0          1          2          3 
% 0.69210344 0.17697177 0.06228221 0.06864257 
% >   dprs=detour.page.rank(sc.sqm,3,d);dprs
%          0          1          2          3 
% 0.79062415 0.07122957 0.05364190 0.08450438 
% >   dprr=detour.page.rank(sc.rcm,3,d);dprr
%          0          1          2          3 
% 0.76538478 0.08355996 0.06247253 0.08858273 %

Another probabilistic analysis can be performed using page rank, which
can be used to compute the probability that a random walk in a graph
visits a certain vertex~\cite{page1999pagerank}. We computed the page rank (with a damping
value of 5\% - meaning that the random walker with probability 5\%
jumps to an arbitrary state and probability 95\% chooses one of the
actions available in the current state) for each vertex in the graph,
and then added the page rank value for vertices with different detour order.
Figure~\ref{fig:pr} shows the resulting probabilities.


\begin{figure}
  \centering
  \includegraphics[width=\linewidth]{figures/pr.png}
  \caption{Probability that a random walk visits detour 0, 1, 2 or 3 states.}
  \label{fig:pr}
\end{figure}

With Gmail the probability of visiting an order 0 state is close to
70\%, the lowest among the four applications. But when it comes to
visiting an order 0 or 1 state, the probability increases to 87\%,
which is the highest. Thus, to compare Gmail with SquirrelMail, a
completely random usage of SquirrelMail has 10\% more probability of
hitting an error-free state than Gmail. That advantage is reduced when
considering order 0 and 1 paths, because with SquirrelMail the
probability is 85\% and Gmail it is 87\%.  This means that somebody
with no knowledge on how to use an email front end, when using Gmail
would have 10\% fewer chances of carrying out the scenario without
making any error, as opposed to when using SquirrelMail: SquirrelMail
provides more guidance. Across the four UIs, the probability of making
at most 1 error is approximately the same\footnote{These results are
  not very sensitive to the value chosen for the damping factor:
  differences across the four applications are very similar when $d$
  ranges in $[0.05,0.20]$.}.

\mytodo{result}

Manual inspection of the shortest paths indicates that one reason of
the higher potential efficiency offered by Gmail is due to the
fact that users can start composing a new message while reading a
conversation, whereas in other applications an explicit ``closing'' of
the reading activity has to be performed. Another reason is in the
more streamlined process to attach a file: in Gmail one needs to
select the file(s) and they are automatically uploaded, whereas in
other applications one has to explicitly perform the uploading step
after selecting them.

\mytodo{result}

\begin{figure}
  \centering
  \includegraphics[width=\linewidth]{figures/ad.png}
  \caption{Action density.}
  \label{fig:ad}
\end{figure}

Figure~\ref{fig:ad} shows the \emph{action density} of the four
applications, in terms of average number of actions per state involved
in traces, and average number of \emph{unique} actions per state. The
former is an overall measure of the number of options that are made
available by a UI, the latter can be used to analyse how many
\emph{new} options the user is presented with in any state. For our
examples, the values are all in the range between 5.9 actions/state in
the case of Gmail and 7.1 for Horde, and 1.3 unique actions for
SquirrelMail and 2 for Horde. This suggests that Gmail features a more
compact design (fewer actions to do the same things), and SquirrelMail
is even more compact when it comes to the different types of actions;
thus it could be easier to learn.

\mytodo{result}

\subsection{Comparing scenarios}

The same kind of analysis can be carried out to assess how suitable a
design is for a set of different scenarios. For example, a designer
might be interested in understanding what are some quantitative
differences in replying to a message as opposed to composing a new
one. 

For Horde, it turns out that composing a new message entails many more
order 0 paths (154 vs 86), and a comparable number of higher order
paths. The path length in the best case is the same across the two
scenarios, but in the average case composing has a length of 7.75
steps compared to 8.5 for reply. Probability of hitting an optimal
path is 4 times higher for compose.


This means that users will have a twice as much large choice of
correct paths when composing a new message rather than ``simply''
replying to a read one. On average, when composing, users could be
9\% more efficient, and they are 4 times more likely to do the right
things. Thus, Horde is more suitable for composing new messages than
it is for replying to existing ones.
\todo[inline]{result}

\subsection{Adding features}

Execution traces can be used also to assess what is the effect of
adding a widget or feature to an existing UI.  For example, we studied
the cruise control features of cars. One of the examples is system S,
where the driver can engage the system, and once it is engaged, speed
can be increased or decreased with small or large steps. Of course the
system can also be disengaged (by pressing the brake pedal, for
example).
System A is more elaborate, as it includes also a memory function:
when it is disengaged it remembers the current speed, which can be
recalled later on.  There are two ways to re-engage it: one
by setting a new speed, and one by recalling the previous one. In
addition,  if the car drives for more than 5 minutes at a higher
speed than the set one, system A automatically disengages.

Thus, one possible design question is ``What are the effects of adding
these functionalities?'' in a typical driving scenario where a speed
is selected, then the system is disengaged, and then the same speed
needs to be set.

System S (where we assumed that re-setting the speed is done manually
by the driver with 4 actions on ``up'' and ``down'' to
approximately get the same speed) leads to 1 order 0 path requiring 7
steps; there are 8 order 1 paths with average length  9.4. The
probability of hitting the optimal path is 0.11; action density is
1.75, and unique action density is 1.25.

On the other hand, system A has 2 order 0 paths (average length 4.5),
and 7 order 1 paths (average length 6.8). 
The optimal path probability is 0.06; action density is
2.2, and unique action density is 1.2.

In both cases the reasons for detours deal with the possibility of
disengaging the system at the wrong moment.

Therefore we can conclude that:
\begin{inparaenum}
\item system A makes users more efficient;
\item with A there are two possible ways to achieve the scenario, thus
  more flexibility is given;
\item with A the probability of doing the right thing is almost half
  of system S: it might be more difficult to do the right thing
  because more possibilities are offered;
\item system A features a more compact design, with fewer action types
  to be performed at each moment, being therefore potentially easier
  to learn and remember.
\end{inparaenum}
\todo[inline]{result}







% \begin{table*}[ht]
% \centering
% \begin{tabular}{rllll}
%   Step &Horde & Gmail & Squirrel & Roundcube\\
%   \hline
%   1 & openMessage(msg) & open(conversation) & openMessage(msg) & openMessage(msg) \\ 
%   2 & exitMessage & goBack & exitMessage &  exitMessage \\ 
%   3 & openMessage(msg) & open(conversation) & openMessage(msg) & openMessage(msg) \\ 
%   4 & reply & reply & reply & reply \\ 
%   5 &  typeBody & typeBody & typeBody                       &  typeBody \\ 
%   6 &  send &  send & send                                  & send \\ 
%   7 &  exitMessage &  & exitMessage                  &  exitMessage \\ 
%   8 & compose &compose & compose                        & compose \\ 
%   9 & addReceiver(receiver) &  addReceiver(receiver)  & addReceiver(receiver)   & addReceiver(receiver) \\ 
%   10 & browseFiles          & attachFiles   & browseFiles  & attachAFile \\ 
%   11 & selectAttachments(file) & open(files)  & selectAttachments(file)          &  selectAttachments(files) \\ 
%   12 & update                &  &add   &  upload \\ 
%   13 &  typeSubject          &  typeSubject          &  typeSubject & typeSubject \\ 
%   14 & typeBody &         typeBody                    & typeBody     & typeBody \\ 
%   15 &  send &           send                    &  send        &  send \\ 
%    \hline
% \end{tabular}
% \end{table*}


\section{Discussion}
\label{sec:discussion}

An important issue underlying \migtool is the modeling effort that is
needed upfront. Our experience, based on several case studies and some
industrial examples, is that models do not need to be complete
representations of the behavior of the application under
study. By following an agile modeling approach~\cite{ambler02}, models can
be easily developed by one person in less than one day, using any UML
capable design tool. Even more complex models (in our experience up to
150 states and 350 transitions) can be developed and verified in 2-3
days by one person.
Experience in using statecharts to model behavior of UIs is needed
though. Useful suggestions are given in~\cite{horrocks99,thimbleby07}.

UML state diagrams provide a very expressive language, well suited to
specify behavior of UIs based on descrete events. Evidence of this can
also be found in recent OMG standards, such as
IFML~\cite{ifml13}. Even though there are fundamental limits
(inability to handle undo/redo's - because that goes beyond a finite
state representation; inability to handle customizable toolbars -
because that requires models that change at runtime; inability to
handle perceptive UIs - because they are not well suited to be
described in terms of descrete states), in many practical cases they
can be be isolated and/or left aside).

Expressivity of the modeling language means that different designers
are likely to produce different models for the same UI. As a
consequence, metrics computed by \migtool do depend on different
modeling choices. It is worth mentioning, though, that because of the
flattening process, several differences are reduced (for example,
those dealing with using a different hierarchy of states, or with
distributing differently concurrent regions across states), and such a
sensitivity is correspondingly reduced. At the moment, however, we do
not have hard evidence to back this claim.

Differently from other model-based approaches, such as those based on
IFML, \migtool uses \emph{only} a model of the behavior of the
UI. Designers do not need to cope with data modeling, nor with
decisions dealing with presentation. In a sense, \migtool uses only the
\emph{controller} part of the Model-View-Controller paradigm that is
adopted when developing UIs. Therefore, the designer using \migtool to
perform analysis is free from other concerns that in the end affect
usability, and the conclusions that are derived with \migtool can be
combined with other results \emph{after} the analysis is
performed. This also means that no effort needs to be directed on
developing data and presentation specifications/implementations.

In terms of validity of conclusions obtained through \migtool, we can
say that because they are devoid of user behavior assumptions (such as
preferences, skills, interpretations, ergonomic constraints) they are
very general. On the other hand, predictions based on \migtool metrics
are also generic because they do not consider data and presentation
aspects, nor user-related factors. For example, it is unfeasible to
use \migtool to predict the time needed by a user to complete a
scenario. However, \migtool can be used to analyze the whole
interaction design and gather data to inform more specific analyses
that could be performed, for example, with CogTool~\cite{john04cogtool}.


\migtool is part of a suite of model-based tools for designing,
analysing and testing user interfaces developed by the company
ANONYMIZED. \migtool is implemented partly in Java (model processing)
and partly in R/iGraph. Development of a web-based UI of \migtool is
underway; \migtool will be made freely available for research
purposes.

\section{Conclusion}
\label{sec:conclusion}





% REFERENCES FORMAT
% References must be the same font size as other body text.
\bibliographystyle{SIGCHI-Reference-Format}
\bibliography{gb,database}

\end{document}

%%% Local Variables:
%%% mode: latex
%%% TeX-master: t
%%% End:
